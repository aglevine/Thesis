\section{LHC}\label{lhc}

\qquad The Large Hadron Collider (LHC) is a proton-proton collider located at CERN in Geneva, Switzerland. The collider was designed to collide protons at a center of mass energy of 14 TeV. It has operated at a center of mass energy of 7 TeV in 2011, 8 TeV in 2012, and 13 TeV in 2015. The following paragraphs will explain how the LHC creates and collides its proton beams.

\qquad The protons used in LHC collisions begin the process as the nuclei of hydrogen atoms in a bottle of hydrogen gas. An electric field is applied, separating the protons and electrons in hydrogen. The protons are then accelerated through Linac 2 (figure ***) up to an energy of 50 MeV. Linac 2 uses radiofrequency (RF) cavities that create alternating electrical fields in conductors. These alternating fields continuously accelerate the protons along the beamline.

\qquad The protons then enter the Proton Synchotron Booster (PSB), which accelerates them up to 1.4 GeV. The PSB accelerates the protons in a circle, using dipole magnets to guide the circular path of the beam and quadropole magnets to focus the beam. The Proton Synchotron (PS) accelerates the protons up to 25 GeV, and finally the Super Proton Synchotron (SPS) accelerates the protons up to 450 GeV, preparing them for their arrival in the LHC.

\qquad After entering the LHC, the protons are further accelerated up to their intended collision energy. The first run, in 2011, accelerated the protons up to 3.5 TeV. This was increased to 4 TeV in 2012, and to 6.5 TeV in 2015. The two proton beams travel in opposite directions around the LHC ring, which is 27 km in circumference. Therefore, the center of mass energy of collisions is twice as much as the energy of the individual beams. The beams are crossed, producing proton-proton collisions at points within each of the major LHC experiments.

\qquad The beams in the LHC are controlled using superconducting NbTi magnets, which are cooled down to below 2K using liquid helium. The magents are designed to produce a magnetic field of 8.3 T, which requires a current of 11850 A. Superconducting magnets are required so that the magnets don't act as resistors against the high current. There are 1232 dipole magnets, which bend the beams in their circular paths, and 392 quadropole magnets, which keep the beams focused.

\qquad The protons in the beam are divided up into "bunches". Each bunch contains $10^{11}$ protons. In 2012, the LHC ran with 1380 bunches per beam, with 50 ns spacing in between bunches. In 2015, the number of bunches per beam was gradually increased up to 2244 bunches per beam, with 25 ns spacing in between bunches. However, the intensity of the beam is lower in 2015 than in 2012. The distance from the focal point where the beam is twice as wide as it is at the focal point ($\beta *$) has increased from 0.6m in 2012 to 0.8m in 2015.

\qquad The performance of the LHC can be measured by how many collisions it creates. The number of events generated per second at the LHC ($N$) is given by: $N = L\sigma$, where $\sigma$ is the cross section (section **) and $L$ is the luminosity. The luminosity is defined as: *** (need to find easier definition). The peak luminosity delivered at the LHC in 2012 was $7 \times 10^{33} s^{-1} cm^{-2}$, and at 2015 it was $5.33 \times 10^{33} s^{-1} cm^{-2}$ 


1e11 protons per bunch, 2835 bunches/beam, 1e34 lumi (at 7 tev, check numbers!!)
check numbers: 4e7 hz bx, 1e9 Hz collisions
1e-13 event selection

transverse emittance ($\epsilon$): smallest opening you can squeeze the beam through
									measurement of the parallelism of a beam
amplitude function $\beta$, $\beta = \pi\sigma^{2}/\epsilon$
where $\sigma$ is cross sectional size of bunch

$\beta^{*}$ is distance from focus point that beam is twice as wide as the focus point

8 TeV (7e33 lumi, f=20e6 Hz, N < 1.7e11 p/bunch, $\epsilon$ < 1.75 um, $\beta^{*} = 0.60m$

13 TeV ($\beta^{*} 0 0.8m$, lumi (peak 5.1e33), 1.1e11 p/bunch, 2244 bunches per beam, 36 bunches/gap/36 bunches

Booster to PS to LINAC2 to SPS to LHC

\section{CMS}\label{cms}
\subsection{Overview}

\qquad The Compact Muon Solenoid (CMS) detector is a hermetic detector at the LHC that is responsible for measuring the energy deposits in an event after the LHC collisions. The LHC beampipes intersect in CMS, allowing for collisions. CMS contains a superconducting magnet which produces a 3.8T magnetic field. This field causes charged particles to bend, and their momentum transverse to the beamline ($p_{T}$) can be identified by using the standard equations for a particle in a magnetic field. The innermost component, the tracker, measures the trajectory of all charged particles in the event. Next, the electromagnetic and hadronic calorimeters measure energy deposits. Muons are highly penetrating particles, so the muon system must reside outside the magnet. The iron yoke in the muon system creates a 2 T magnetic field, in conjunction with CMS's magnet, enabling muon tracks to be measured. 

\qquad A diagram of CMS's geometry is in figure ****. The $z$ direction points along the beamline and the $x-y$ coordinate system is perpendicular to the beamline. The azimuthal angle in the $x-y$ plane is $\phi$ and the polar angle is $\theta$. Instead of the polar angle, the lorentz invariant quantity $\eta$ (pseudorapidity) is used, where $\eta = ln(tan(\theta/2))$.


\subsection{Tracker}\label{tracker}
silicon sensors (strips and pixels), measure momenta, identify vertex
pixel: n+ in n bulk, 100x150 mum^2
at r = 4.4, 60 MHz/cm2 at peak lhc lumi 1e34
3e14 n/cm2/yr
10-4 occupancy

18m and 48m barrel pixels
66 million pixels
10 million strips
pixels seed tracking, vertexing, 95\% efficiency
strip tracker: design occupancy 1-2\%
outer cell size 20x100-200 um
inner cell sizer 10X80 um

petals: intermediate structure
400 pieces makes one petal, needed 288 total

readout via optical fibers

electron hole pair pairs generated by particles are separated by e field and drift to electrodes

strips: wire bonds, multiplxer
pixels:SiN strips, readout chips, pixel sensor, bump bonding, highdensity interconnect, token bit manager
strips: synchronous, pixel: asynchronous

pixel detector occupancy at least one order of magnitude smaller than strips

talk about distribution of material with respect to eta

barrel pixels can move over time
movements checked on runs > 20k events
significant movement-> alignment recomputed on 100k event run

phase II upgrade (from 2014 slides) more radiation tolerant, finer granularity, lower mass, forward to eta4, l1 trigger capability
\subsection{ECAL}\label{ecal}
barrle, endcap, preshower

lead tungstate crystals

8.28 g/cm^3, 0.85 radiation length, 2.19 moliere radius, 440 nm wavelength peak, <10 fast decay constant, 100 light yield per MeV

em shower generated by electron/photon, charged particles produce scintialltion light isotropically, 
proportional to particle energy, detected by photodetectors with internal amplification (EB- avalanche photodiodes), (EE- Vacuum PhotoTriodes)

PbWO4 crystals transparent to entire scintillation spectrum before irradiation

barrel: eta < 1.48
	36 supermodules: 1700 crystals (1 supermodule = 4 modules)
	61200 crystals total, 17 shapes
	2.2x2.2x23 ~26 radiation lengths
endcaps: 1.48 < eta < 3.0
		 4 disc (2 per endcap) 3662 crystals (mostly in 5x5 supercrystals)
		 14648 crystals total of 1 shape
		 3.0 x 3.0 x 22 ~25 Xo
preshower 1.65 < eta < 2.65
		 4 planes (2 per endcap) 1072 silicon sensors
		 1 sensor = 6.3 x 6.3 x 0.032 cm^3, 32 strips, 137216 strips total
		 2Xo + 1Xo of Pb with Si strips
		 1.90 x 61 mm2 x-y view
		 
front end electronic: mgpa with 3 separate gains
- 1 mgpa + 1 adc per crystal
fenix chip stores digital signal from multiple crystals until level 1 trigger signal received
another fenix chip sums energy in group of crystals (5x5 in EB, 5 in EE, tpg sent to trigger electronics)
sum energy for trigger at 40 MHz, crystal data sent at 100 kHz
multi gain pre amplifier 

energy resolution 
$\frac{\sigma_{E}}{E} = \frac{A}{\sqrt{E}} \oplus \frac{B}{E} \oplus C$
A = stochastic term (energy fluctuations)
B = noise term (quantum electronic/ pu noise)
C = constant term (contruction, stability, uniformity)

A = 2.8\%
B = 0.128
C = 0.3

energy resolution based on 3x3 arrarys of barrel crystals

L1: 2012, 5x5 towers for egamma

l1 eg trigger 2012: 
5x5 crystals in barrel
4 l1 eg candidates per region (4x4 towers)
keep 4 candidates with highest et in ECAL
fine grain veto: 90\% tower Et within 2 adjacent strips
H/E < 5\%
isolated stream: sum of 5 adjacent towers < 3.5 GeV,8 neighbor towers pass FG and H/E selections
non isolated stream

laser calibrations: 447 nm, averaged over 40 minutes for each crystal

preshower: 6.3 x 6.3 cm, 1.9 mm strips Si sensors
vetoes pi0 faking single photons by resolving closely spaced incident photons

\subsection{HCAL}\label{hcal}
outer, barrle, endcap, forward, castor, zdc

incident hadron generates em shower in brass absorber, scintillation light produced in plastic,
transported to heavy brass absorber

copper plates (70\% copper, 30\% zinc)
copper plates "interleaved" with plastic scintillators with wavelength shifting fibers
light channeled using clear fibers to photodetectors located at end of barrel
HF: radiation hardness important, stell plates with cherenkov producing quartz fibers

barrel: eta < 1.3, 36 wedges ( 18 +/-)
14 layers of brass: 10 interaction lengths
16 megatildes, 16 eta and 4 phi division per wedge

endcaps: 1.3 < eta < 3.0, 36 wedges per endcap
17 layers of brass: 10 interaction lengths
17 megatile layers: 12 eta and 1-2 phi division per wedge
2-3 high eta longitudinal segments

forward: 3.0 < eta < 5.0, 18 wedges per end
steel plates 5mm thick, 165 cm 10 interaction lengths
grid of holes 5mm apart
1mm diameter fibers (600 um quartz core + cladding + buffer)
half are 165 cm long
other half start after a depth of 22 cm

castor:
5.2 < eta < 6.6 on - side of cms 
14.38 from interaction point, layers of tungsten plates interleaved with fused silican quartz plates
light readout via pmts
em (20.1 x0, .77lambda), hadronic (9.24)

zdc:
heavy ion physics , eta > 8.3
tungsten plates and quartz fibers

readout: hybrid photo-detectors, designed to operate in high magnetic field
proximity focusing with 3.5 mm gap, with e field || to B field
gain of 2k, linear response over large dynamic range from mips up t o3 TeV hadron showers

charge integrator and encoder (QIE) ADC chips digitze HPD/PMT signals
QIE has 4 capacitors, connected to input by 25 ns time intervals
ever 25 ns integrated charge from capacitors is sent to HCAL trigger boards

about 90\% energy response above 20 GeV track momentum



HF lumi measurement
\subsection{Muon System}\label{muonsys}
dt's and csc's: positions measurement
rpc's: redundant triggering
0< eta < 1.2: 5 wheels, 4 stations, instrumented with DT's and RPC's
endcap: 0.9 < eta < 2.4: 3 discs, 4 stations, instrumented with csc's and rpc's

gas detectors to cover a large surface area
amplify signal within gas volume

drift tubes
250 (or 240?) chambers in barrel
drift time measurement to few ns gives 250 mum accuracy

12 layers per chamber

argon(85) + co2(15)

csc's: 540 in edncaps, cathode strips gives 200 um accuracy
6 layers
precision phi from cathode strips, rougher r from anode wires

rpc's: 480 in barrel, 576 in endcap
double gap, each 2mm 9.6

detectors located beyond 10 interaction lengths, little punchtrough of hadrons

punchthrough rate dominated by particles made in hadronic showers

muons come to rest before 4th muon station if p<5.2 gev (pt < 1.9 at eta=2)

muon trigger efficiency: dips due to crack between dt wheels 0 and pm1 in eta 0.2-0.3
eta > 1.2 assymmetry due to csc non operational chambesr ***

spatial resolution: 80-120 um in dt's 40-150 um in csc, 0.8-1.2 cm in rpc's

efficiency 95-98\%
\subsection{Trigger}\label{trigger}

reduce 40 MHz to 1 kHz
l1 trigger turns 40 MHz into 100 kHz
HLT turns 100 kHz into 1 kHz

fast readout from detector
FPGA and ASIC hardware 

include diagram of L1 trigger

HF, HCAL, ECAL send tpgs to RCT
RCT sends e/gamma, region Et to global calo trigger
4 iso e/gamma, 4 e/gamma, 4+4 jet, 4 tau, Et, Ht
RPC sends to pattern comparator
DT to DT local to DT trackfind/ CSC trackfind
CSC to CSC local, CSC trackfinder
RPC pattern comparator, DT trackfinder, CSC trackfind to GMT
GMT sends 4mu to GT


DT and CSC: DTTF and CSCTF build muon candidates from tracks
	each candidate assigned eta, phi, pt, quality
	select 4 DT and 4 CSC candidates

HCAL primitives
	ET in each tower
	
e/gamma candidates: id based on shower shape, isolation from ECAL and H/E
	4 isolated and 4 non-isolated e/gamma candidates
	
jet candidates:
	calorimeter regions (4x4 towers)
	sum et of 3x3 regions
	4 central jets (eta < 3), 4 forward jets (eta > 3)
	4 tau jets (eta < 3)
	
Et, Ht, MET MHT, computed from all regions above threshold

L1 global trigger reads candidates from muon and calo triggers
defines up to 128 algorithms

L1 Trigger upgrade
replace L1 GCT with Stage 1 Layer 2
dedicated trigger for taus
*include performance plot*
discuss tau veto bit, 

DAQ:
detector readout
event builder
HLT farm, cluster filesystem

HLT
200 ms average time to make a decision
400 Hz average output rate
runs on farm of commercial computers
13,000 cpu cores, 20000 processes

L2 standalone muons
L3 global and tracker muons
tracker based isolation

photons: ecal superclusters
calorimeter based id and isolation: tracker id and isolation

taus: particle flow reconstruction

jets, MET, HT: calorimetric jets and MET, particle flow based jets and MET

bjets: jets, full tracking, secondary vertex reconstruction

collects streams for physics analysis, trigger studies, calibration workflows
